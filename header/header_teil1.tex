%%%%%%%%%%%%%%%%%%%%%%%%% Standardpakete %%%%%%%%%%%%%%%%%%%%%%%%%
\usepackage[utf8]{inputenc}        % Eingabe von Umlauten im Code
\usepackage[T1]{fontenc}           % Enkodierung von Output, böse Dinge passieren ohne dieses Package
\usepackage[ngerman]{babel}        % Deutsches Sprachpaket
\usepackage{scrhack}               % Fixt Probelem mit KOMA-Skript
\usepackage{mathtools}             % Mathematische Formeln
\usepackage{microtype}             % Macht irgendwie typographisch besseren Text
\usepackage{graphicx}              % Hinzufügen von Bildern / PDFs über \includegraphics{bildname}
\usepackage{caption}               % Erweiterte Captions für Bilder, Formeln, ...
\usepackage{physics}               % Zusätzliche wissenschaftliche Symbole und Differentialquotient mit \dd{...}
\usepackage{hyperref}              % Links und Verweise in PDFs
\usepackage{siunitx}               % Wissenschaftlich korrektes Angeben von Zahlen / Einheiten (wichtig bei Hampe, Siaenen), u.A. Einheiten nicht kursiv
\usepackage{listings}              % Includen von Code
\usepackage{xcolor}                % Einfach eigene Farben definen
\usepackage{url}                   % Includenb von Links mit \url{http://thomas-harriehausen.de}             

%%%%%%%%%%%%%%%%%%%%%%%%% Optionale Pakete, die auch entfernt werden können %%%%%%%%%%%%%%%%%%%%%%%%%
\usepackage{lastpage}              % Erlaubt "Seite 2 von x" im Footer, da es die Anzahl Seiten ausgibt
\usepackage{float}                 % Welches dann in Kombination mit [H] auch wirklich die Gleitumgebung mit dem Bild oder der Tabelle an die richtige Stelle packt, nämlich (H)ere 
\usepackage{array}                 % Matrizen in Latex
\usepackage{tabularx}              % Tabellen, die advanced sind
\usepackage{enumitem}              % Erweiterte Aufzählungen, bespielsweise alternative Aufzählungssymbole 
\usepackage{gensymb}               % Für Grad-Symbol \degree, Ohm: \ohm und einige mehr
\usepackage{esvect}                % Vektorpfeile mittels \vv{E}
\usepackage{icomma}                % Richtgen Abstand beim Dezimaltrennzeichen
\usepackage[all]{nowidow}          % Vermeidet "widows" (einzelne Zeilen auf nächster Seite die noch zum Absatz gehören)
\usepackage{chngcntr}              % Ermöglicht Ändern von Nummerierung (Formeln, Bilder), z.B. Zählen innerhalb Kapitel
\usepackage{subcaption}            % Bilder nebeneinander darstellen
\usepackage{csquotes}              % Für Zitate und um Warnung von babel zu unterdrücken
\usepackage[draft]{todonotes}      % \todo[inline]{beispiel}-Befehl, der Kommentare hinzufügt. Ausblenden der Todos durch ersetzen von "draft" durch "disable"
\usepackage{multirow}              % Für Tabellen: Mehrere Zeilen in einer Spalte miteinander verbinden
\usepackage{booktabs}              % Für das Excel2LaTeX-Addon, s.u. in Tipps
\usepackage{bigstrut}              % Für das Excel2LaTeX-Addon, s.u. in Tipps
\usepackage{cancel}                % Für durchgestrichenen Text (schlechte Messwerte zB) mit \cancel{text}
\usepackage{easyfig}               % Einfaches includen von Bildern z.B.: \Figure[caption={LOL}, width=\textwidth]{dateiname} (Labels sind automatisch \autoref{fig:dateiname})
\usepackage{tcolorbox}             % Für colorboxes (genutzt für Titelseite)
\usepackage{pgfplots}              % Um Diagramme direkt mit Latex zu erstellen
\usepackage{xifthen}               % Für if else Logic, wird im Header benutzt
\usepackage{pdfpages}              % Um PDF Dokument hinzuzufügen
\usepackage{makecell}			   % ermögtlicht einfache Zeilenumbrüche in einer tabelle
\usepackage{longtable}			   % tabellen über mehere Seiten mit advanced funktionen
\usepackage{amssymb}				% viele weitere symbole
\usepackage{stix2}					% viele mathematische symbole
%\usepackage{ulem}					% druchstreichen
\usepackage[normalem]{ulem}			% dursctreichen jedoch ohne Quellenprobleme
\usepackage{trfsigns}				% weitere besondere Symbole bspw. Laplace
%%%%%%%%%%%%%%%%%%%%%%%%% Einstellung für Pakete und LaTeX %%%%%%%%%%%%%%%%%%%%%%%%%
% Bestimmte Warnung von chktex (Latex linter) deaktivieren:
% chktex-file 46
% chktex-file 2
% chktex-file 1
% chktex-file 41
% chktex-file 10
% chktex-file 17

% Language Tool in vs code auf deutsch stellen (Extension LTex installieren):
%LTeX: language=de-DE

% Einstellen von Seitenabständen und Größen
\usepackage[paper=a4paper, 
left=25mm,                    
right=20mm,                   
top=15mm,                     
bottom=15mm,                  
includehead,includefoot            % Fuß- und Kopfzeile berücksichtigen
]{geometry}

% Schriftart ändern
\usepackage[space]{erewhon} % Main Schriftart 1 
%\usepackage{stix2} % Main Schriftart 2
\usepackage{tgheros} % Serifenlose Schriftart, wird deshalb nur für Überschriften genutzt
%\usepackage[stix2,vvarbb]{newtxmath} % Nur sinnvoll wenn aktiverte Schrifart keine (schönen) Math-Symbole hat
% Optional: Serifenlose Schriftart aktivieren um Leute zu triggern
%\renewcommand{\familydefault}{\sfdefault}

\setcounter{secnumdepth}{6} % tiefere referenz ebene möglich z.b. paragraph \label{key}

% Automatisch Text wie "Zitat" in deutsche Anführungszeichen umwandeln
\MakeOuterQuote{"}

% Warnung von pgfplot unterdrücken
\pgfplotsset{compat=1.7}

% Einstelllungen für das siuntix-Paket
\sisetup{
	per-mode=fraction,             % Einheiten als Brüche statt ^-1
	locale=DE,                     % Malpunkt, statt Kreuz
	output-decimal-marker={,},     % Komma als Dezimaltrennzeichen
	group-digits=true,             % Zifferngruppierung an/aus (in 3er Blöcken)
	group-separator=\text{~},      % Abstand als Trennzeichen für Zifferngruppierung
	group-minimum-digits=5,        % Ziffern ab minimal 5 Ziffern gesamt gruppieren
	detect-all,                    % Benutze gleiche Schriftarten wie im Text
	range-phrase=--,               % Komma in Range
	range-units=single,             % Nur einmal die Einheit bei Range
	exponent-to-prefix = true,	% bei eingabe von \SI{1.7e3}{\g} kommt 1,7 kg raus 
	%	exponent-to-prefix = false,		% bei eingabe von \SI{1.7e3}{\g} kommt 1,7\cdot 10^3 g raus 	
	%	scientific-notation = engineering, % nimmt die exponenten nach der Ingienuerschreibwese ! Achtung bei der S umgebung der Tablle mit table-format !
	scientific-notation = false,
	output-complex-root = j,		 % anstelle i j ausgeben
	complex-root-position = before-number, % stellt das j vor der zahl
	exponent-product = \cdot
} 

%Die Umgebung von TableS is gleich zu Table, allerdings wird scientific-notation ausgeschaltet innerhalb der Umgebung. Falls Fehler wie: "Package siunitx Error: No space reserved for an exponent" immer tableS statt table verwenden!
\newenvironment{tableS}{\sisetup{scientific-notation = false}  \begin{table}}{\end{table}\sisetup{scientific-notation = engineering} }

\DeclareSIUnit\divisor{div.}       % SIunitx: Volt pro Divisor, zB bei Schirmbildern      
\DeclareSIUnit\voltpeak{Vp}        % Für Vp in SI
\DeclareSIUnit\voltppeak{Vpp}      % Für Vpp in SI    

% Biblatex includen für Zitieren
\usepackage[style=numeric,         % Zitieren mit [1] statt [Ein05]
sorting=none,                      % Referenzen sotieren nach Ort des Auftauchen
backend=bibtex                     % Unterdrückt irgendeine Warnung
]{biblatex} 
%\addbibresource{../sources.bib}       % Optional, fügt sources.bib aus gleichem Verzeichnis hinzu
%\addbibresource{../../sources.bib}       % Optional, fügt sources.bib aus gleichem Verzeichnis hinzu
\setcounter{biburllcpenalty}{7000} % für lange URLs im Quellverzeichnis
\setcounter{biburlucpenalty}{8000} %

% Hinzufügen von Fuß- und Kopfzeilen mit Trennlinie
\usepackage[
headsepline,
footsepline,
automark
]{scrlayer-scrpage}   

% circutikz-Paket: "Zeichnen" von Schaltungen im Code
\usepackage[european,              % Europäischer Style für ciruitikz
straightvoltages,                  % Gerade Zählpfeile in DE
americaninductors,                 % Meist wird alte Norm mit Ami-Spulen genutzt
siunitx,                           % Integration von siunitx aktivieren
nooldvoltagedirection              % Neue Art für Spannungsrichtung
]{circuitikz}                      

% Quellen ab x starten
%\newcommand{\quellenstart}{3} 
%\DeclareFieldFormat{labelnumber}{
%    \ifinteger{#1}
%    {\number\numexpr#1+\quellenstart\relax}
%    {#1}}

% Zählen innerhalb von Kapitel für Bilder / Gleichungen: <section>.<number>
\counterwithin{figure}{section}
\counterwithin{table}{section}
\counterwithin{equation}{section}

% Bilder in diesen Subfoldern müssen den Folder beim includen nicht mehr im Dateinamen angeben
\graphicspath{{../fig/}{../bilder/}{../Bilder/}{../images/}{../figs/}{../../fig/}{../../bilder/}{../../Bilder/}{../../images/}{../../figs/}}

% Höherer Zeilenabstand, Dirty Harrie will eigentlich 1.2, ist außerdem gut, weil's nach mehr Seiten aussieht
\linespread{1.15}

% Latex sorgt dafür, dass linebreaks stärker erzwungen werden und nicht Wörter über den Rand hinausragen; siehe https://texfaq.org/FAQ-overfull
\tolerance=7500
\pretolerance=200

% Etwas mehr Abstand zwischen den einzelnen aligns
\addtolength{\jot}{0.4em}

% Macht vertikalen Abstand in Tabellen größer, dieser ist default sehr eng
\renewcommand{\arraystretch}{1.15} 

% Kein Einschub bei Auflistungen, Geschmackssache
\setlist[enumerate]{leftmargin=*}

% Aufzählungen mit a) b) statt 1. 2.
\renewcommand{\labelenumi}{\alph{enumi})}

% Ersetzt alle * im Math-Environment durch richtige Malzeichen (cdot). Wenn * für konj. komplex oder ähnliches genutzt werden soll, dann auskommentieren
%\mathcode`\*="8000
%{\catcode`\*\active\gdef*{\cdot}}

% Indizes standardmäßig NICHT kursiv, da es nach DIN-Norm eher in Ausnahmen kursiv ist. Hilfreich für MET- und EMT-Labor ausnahme mit \mathit{}
% bsp : \mathit{A_X}  -> Ausgabe ist A kursiv indizi X nicht kursiv 
\makeatletter
\begingroup
\catcode`\_=\active
\protected\gdef_{\@ifnextchar|\subtextit\subtextup }
\endgroup
\def\subtextit|#1|{\sb{#1}}
\def\subtextup#1{\sb{\mathrm{#1}}}
\AtBeginDocument{\catcode`\_=12 \mathcode`\_=32768}
\makeatother

% Gleichung in Formel umbennen, Harrie mag das lieber
\renewcommand{\equationautorefname}{Formel} %% not working

% Für das todonotes package
\reversemarginpar                  % Randnotizen auf der linken Seite, da dort mehr Platz ist
\setlength{\marginparwidth}{2cm}   % Randnotizen-Breite festlegen, da das Paket sonst nicht funktioniert 

% Einige vordefinierte Colors
\definecolor{listingbg}{cmyk}{0,0,0,0.05}
\definecolor{ostfalia-magenta}{cmyk}{0,1,0,0}
\definecolor{ostfalia-green}{cmyk}{0.3,0,1,0}
\definecolor{ostfalia-cyan}{cmyk}{0.78,0,0.32,0}
\definecolor{ostfalia-orange}{cmyk}{0,0.3,1,0}
\definecolor{ostfalia-yellow}{cmyk}{0,0.05,1,0}
\definecolor{ostfalia-violet}{cmyk}{0.86,0.96,0,0}
\definecolor{ostfalia-wf-blue}{cmyk}{1,0,0,0}
\definecolor{ostfalia-blue}{cmyk}{1,0.75,0,0.3}

% Eigenes Aussehen für Code-Blöcke definieren, kopiert von 
% https://github.com/Wandmalfarbe/pandoc-latex-template
\definecolor{listing-background}{HTML}{F7F7F7}
\definecolor{listing-rule}{HTML}{B3B2B3}
\definecolor{listing-numbers}{HTML}{B3B2B3}
\definecolor{listing-text-color}{HTML}{000000}
\definecolor{listing-keyword}{HTML}{435489}
\definecolor{listing-keyword-2}{HTML}{1284CA} % additional keywords
\definecolor{listing-keyword-3}{HTML}{9137CB} % additional keywords
\definecolor{listing-identifier}{HTML}{435489}
\definecolor{listing-string}{HTML}{00999A}
\definecolor{listing-comment}{HTML}{8E8E8E}

%%%%%%%%%%%%%%%%%%%%%%%%% Eigene Befehle %%%%%%%%%%%%%%%%%%%%%%%%%
% Kleines Underline, was sich gut für komplexe Zahlen eignet, z.B. \ul{U}_q
\newcommand{\ul}[1]{\underline{#1\mkern-1.5mu}\mkern 1.5mu}
% Text hervorheben als code / mono
\newcommand{\code}[1]{\ttfamily#1\rmfamily} % Beispiel: \code{AC}-Modus
% \desc{} Command für Text in Math-Mode mit Abstand vor und nach dem Text, nützlich um zB in align etwas beschreiben. z.B.: \desc{eingesetzt ergibt dies:}
\newcommand{\desc}[1]{{\hspace*{0,7 cm}\text{#1}\hspace*{0,7 cm}}}

\newcommand{\paraNumOFF}{
	\renewcommand{\theparagraph}{\thesection.\arabic{subsection}.\arabic{subsubsection} \alph{paragraph})}
} % aufzählung im paragraph mit 1.1.2 a). 1.1.2 b). und 1.1.2 c).

\newcommand{\paraNumON}{
	\renewcommand{\theparagraph}{\thesection.\arabic{subsection}.\arabic{subsubsection}.\arabic{paragraph}}
} % aufzählung bei paragraph mit 1.1.2.1 1.2.2 und 1.1.2.3


\newcommand{\subsubSecNumOFF}{
	\renewcommand{\thesubsubsection}{\thesection.\arabic{subsection} \alph{subsubsection})}
} % aufzählung im subsubsectionmit 1.1 a). 1.1 b). und 1.1 c).

\newcommand{\subsubSecNumON}{
	\renewcommand{\thesubsubsection}{\thesection.\arabic{subsection}.\arabic{subsubsection}}
} % aufzählung bei subsubsectionmit 1.1.1 1.2.2 und 1.1.3


% Buchstaben vor Subsections für Laborberichte (V2.2, A3.1, usw.) Benutzung mit \laborsubsection{V}{Überschriftstext}
\newcommand{\laborsubsection}[2] {
	\renewcommand{\thesubsection}{#1 \thesection.\arabic{subsection}}
	\subsection{#2}
	\renewcommand{\thesubsection}{\thesection.\arabic{subsection}}
}

% Buchstaben vor Subsubsections für Laborberichte (V2.2, A3.1, usw.) Benutzung mit \laborsubsubsection{V}{Überschriftstext}
\newcommand{\laborsubsubsectionAlpha}[2] {
	\renewcommand{\thesubsubsection}{#1 \thesection.\arabic{subsection} \alph{subsubsection})  }
	\subsubsection{{\boldmath #2}}
	\renewcommand{\thesubsubsection}{\thesection.\arabic{subsubsection}}
}

% Einfacher Befehl, um Subsection wieder bei 1 starten, damit Wechsel von V 2.2 zu D 2.1 möglich ist
\newcommand{\resetlaborsectioncounter}{\setcounter{subsection}{0}}
\newcommand{\setlaborsectioncounter}[1]{\setcounter{subsection}{#1-1}}

% Helfer-Makro, wird weiter unten im Header benutzt.
\newcommand{\ifempty}[3]{\ifthenelse{\equal{#1}{}}{#2}{#3}}

% Abkürzungen mit korrekten Typographie (achtung Autismus)
\newcommand*{\zb}{z.\,B.~\allowbreak}
\newcommand*{\ua}{u.\,a.~\allowbreak}

%Euler Form (e^j)
\newcommand{\keulerg}[2][]{\mathrm{e}^{\mathrm{#1j}\,#2^\circ}} % komplexe zahl in euler form mit grad
\newcommand{\keulerk}[2][]{\mathrm{e}^{\mathrm{#1j}\,\left( #2\right) }} % komplexe zahl in euler form ohne grad mit klammern
\newcommand{\keuler}[2][]{\mathrm{e}^{\mathrm{#1j}\,#2}} % komplexe zahl in euler form ohne grad ohne klammern 
\newcommand{\dreiVec}[3]{\left( \begin{matrix}
		#1\\
		#2 \\
		#3 \\
	\end{matrix}\right) } % einfacher 3er vector mit einem Befehl \dreiVec{1}{2}{3} !nur in mathe umgebnungen ! 

\newcommand{\fancypara}[2][0.5ex]{\paragraph{{\boldmath #2 }} $ $ \\[#1]}   % pragraph welcher alles dick schreibt und man den Abstand zur neuen Zeile einstellen kann (standart abstand ist 0.5em).
\newcommand{\dummyLorem}{
	Lorem ipsum dolor sit amet, consectetur adipiscing elit. Quisque nec consectetur lorem. Aliquam diam nisi, vestibulum et eleifend mollis, luctus sit amet diam. Donec in magna non enim lobortis ultricies eget ac eros. Etiam eget enim pretium arcu sagittis porttitor ac eu tellus. Curabitur quis cursus lorem. Aliquam auctor est auctor metus lobortis convallis. Maecenas elementum sem sed sem suscipit, a mattis justo volutpat. Praesent et pharetra ex. Morbi ac eros quam. Nullam id libero sit amet dolor aliquet tincidunt. Suspendisse scelerisque eleifend tristique. Curabitur nec nisi id arcu tempus malesuada vehicula at enim. } %dumm text
\newcommand{\dummy}{
	Text Text Text Text\\
	Text Text Text Text\\
	Text Text Text Text\\} 
% Eigene Daten, werden dann in PDF übernommen
\newcommand{\myAuthor}{\myAuthorA, \AutorAMartikelnummer} 
\newcommand{\mySecondAuthor}{\myAuthorB, \AutorBMartikelnummer}
\newcommand{\myDate}{\today} % \today kann durch Text ersetzt werden

\newcommand{\rem}[1]{\textcolor{red}{\sout{#1}}} % ein befehl um sachen zu streichen (hilfreich bei teamarbeit)
\newcommand{\add}[1]{\textcolor{blue}{#1}} % ein befehl um deinen kollegen zu seiten was neu ist
% nutzung von \kont 
%\kont[0]{add argument}{rem argument}    % ich zeige beides mit farbe an wie ohne 0
%\kont[1]{add argument}{rem argument}    % ich behalte den Inhalt des erstes arguments (add) und schreibe das ohne farbe oder durchstreichung  
%\kont[2]{add argument}{rem argument}    % ich behalte den Inhalt des zweiten arguments (rem) und schreibe das ohne farbe oder durchstreichung
%\kont[...]{}{rem argument}    % funktinoniet wie rem
%\kont[...]{add argument}{}    % funktinoniet wie rem
\newcommand{\kont}[3][]{%
	\ifthenelse{\equal{#1}{1}}{#2}{%
		\ifthenelse{\equal{#1}{2}}{#3}{%
			\ifthenelse{\equal{#1}{0}}{%
				\textcolor{blue}{#2} \textcolor{red}{\sout{#3}}}{%
				\textcolor{blue}{#2} \textcolor{red}{\sout{#3}}}}}%
}
\newcommand{\kommentar}[2][]{
	\ifthenelse{\equal{#1}{1}}{
		\colorbox{green}{\textcolor{orange}{#2}}
	}{%
	}
}